\documentclass[12pt]{article}

\usepackage[utf8]{inputenc}
\usepackage{geometry}
\usepackage{graphicx}
\usepackage{titling}
\usepackage[colorlinks=false, linkcolor=blue, urlcolor=blue, citecolor=blue]{hyperref}
\geometry{margin=1in}
\usepackage{caption}
\captionsetup{justification=centering,font=scriptsize}


% --- Remove default title spacing ---
\setlength{\droptitle}{-2em}  % adjust vertical spacing if needed

% --- Title and author manually ---
\title{\huge\bfseries Temporal Dynamics and Volatility Analysis of the PHP--USD Exchange Rate}
\author{
\footnotesize Baul, K.L., Ca\~{n}ete, M.C., Galario, A., Lapad, G.M., Panilag, A.J., Ramirez, G.M.\\[0.3em]
\footnotesize Department of Data Science\\
\footnotesize University of Science and Technology of Southern Philippines -- CDO
}

\date{}

\begin{document}

% --- Custom title layout ---
\begin{center}
    \includegraphics[width=0.16\textwidth]{logo.png}\par\vspace{1em}
    {\huge\bfseries Temporal Dynamics and Volatility Analysis of the PHP--USD Exchange Rate}\par\vspace{1em}
    {\footnotesize
    Baul, K.L., Ca\~{n}ete, M.C., Galario, A., Lapad, G.M., Panilag, A.J., Ramirez, G.M.\\
    Department of Data Science\\
    University of Science and Technology of Southern Philippines -- CDO\\[0.5em]
    }
\end{center}

\begin{center}
\textbf{Abstract}
\end{center}

\
This study analyzes the behavior, volatility, and structural shifts of the PHP–USD exchange rate from 2018 to 2025. The exchange rate is treated as a dynamic indicator of broader macroeconomic conditions rather than an isolated variable. The analysis integrates inflation and interest rate differentials between the Philippines and the United States, along with trade balance measures, oil prices, and Overseas Filipino Worker (OFW) remittances. Daily USD–PHP closing prices serve as the primary dataset, while monthly macroeconomic indicators are temporally aligned through resampling and feature engineering.

Descriptive statistics, rolling-window measures, and visual time-series techniques are employed to examine trends, volatility patterns, and co-movements without imposing strict econometric assumptions. The results identify distinct phases of exchange rate behavior marked by stability, heightened volatility, and sharp directional changes. Comparative inflation analysis reveals higher variance, faster mean reversion, and weaker persistence in Philippine inflation relative to the smoother and more stable U.S. inflation process. Overall, the findings underscore the sensitivity of the PHP–USD exchange rate to relative macroeconomic pressures and structural dynamics, offering insights relevant to macroeconomic monitoring and risk analysis.

\section{Introduction}

\subsection{Background and Significance}

The PHP--USD exchange rate is a preeminent macroeconomic indicator for the Philippines, a small, open economy heavily reliant on remittances and imports. Fluctuations in the exchange rate directly affect domestic inflation—particularly through the cost of imported commodities such as oil—fiscal stability via US dollar--denominated debt servicing, and the welfare of remittance-receiving households.

Given these critical implications, understanding exchange rate dynamics requires moving beyond a univariate time-series perspective. Classical currency valuation theories, notably Purchasing Power Parity (PPP) and Interest Rate Parity (IRP), posit that exchange rates are fundamentally influenced by cross-country differentials in inflation and interest rates. Consistent with these frameworks, this study incorporates Philippine inflation, US inflation, and the US Federal Funds Rate (DFF) to examine their relationship with the PHP--USD exchange rate.

The 2018–2025 period is particularly salient, as it encompasses extreme macro-financial shocks, including the COVID-19 pandemic (2020) and the aggressive US Federal Reserve interest rate tightening cycle (2022–2023). These events generated pronounced monetary policy divergence, significantly shaping capital flows and exchange rate volatility. Using a multivariate exploratory data analysis (EDA), this study investigates co-movement patterns and volatility dynamics among these variables, providing empirical insights relevant to exchange rate policy and financial risk management.

\subsection{Statement of the Problem}

From 2018 to 2025, the USD–PHP exchange rate experienced significant fluctuations driven by major global and domestic events, including the COVID‑19 pandemic, shifts in interest rates and inflation, oil price instability, and geopolitical tensions such as the Russia–Ukraine war. These events likely affected currency behavior, but the extent, patterns, and timing of their influence remain unclear. This project aims to conduct exploratory data analysis (EDA) to examine how the USD–PHP exchange rate behaved during this period, identify potential seasonal trends, measure changes in volatility before and after the pandemic, and assess whether major events or macroeconomic factors contributed to notable movements in the exchange rate. Understanding these patterns can provide insights into currency sensitivity to shocks and support more informed economic analysis.

\section{Methodology}

\subsection{Primary Datasets}

This research employs the USD--PHP exchange rate as its primary time-series dataset. We sourced the USD--PHP exchange rate dataset from \href{https://www.investing.com/currencies/usd-php}{Investing.com (2025)}. To maintain a parsimonious model and ensure data stationarity, we extracted only the Daily Closing Price, omitting the auxiliary Open, High, and Low (OHL) metrics. The rationale for this selection is twofold: first, the closing price represents the aggregate daily market consensus; and second, it minimizes volatility bias inherent in intraday trading. This allows for a more robust correlation analysis with secondary datasets that lack high-frequency, intraday granularity.

\subsection{Secondary Datasets}

To provide a comprehensive analysis of the USD-PHP exchange rate, this study integrates a primary exchange rate dataset with four distinct categories of secondary macroeconomic variables. These variables represent the fundamental drivers of currency valuation, encompassing both financial flows and real-economy demands.

First, the study incorporates Interest Rates and Inflation Rates from both the United States and the Philippines. These are utilized to capture the 'interest rate differential' and 'purchasing power parity' effects, which dictate capital allocation and investment flows between the two nations.

Second, to account for the structural supply-demand dynamics of the Philippine economy, the study integrates Merchandise Trade Statistics—specifically the total value of Merchandise Imports and Exports for the Philippines. In the Philippine context, the Merchandise Trade Balance is a critical determinant of currency value; a widening trade deficit in physical goods necessitates a higher demand for US Dollars to settle international payments, typically leading to Peso depreciation. By focusing on merchandise trade, the analysis captures the direct impact of the physical economy on foreign exchange requirements.

Furthermore, as the Philippines is a net energy importer, international Oil Prices (Crude and Brent) are included to reflect the cost-push inflationary pressures and the subsequent drain on foreign exchange reserves required for fuel procurement. Finally, OFW Remittances are integrated as a primary supply-side factor. As a significant contributor to the Philippine GDP, these remittances represent a consistent inflow of US Dollars, acting as a key stabilizer for the local currency against external shocks. By combining these datasets, the model can effectively distinguish between short-term financial speculation and the fundamental commercial requirements of the Philippine economy.

\subsection{Data Cleaning and Preprocessing}

\subsubsection*{Data preparation}
\begin{itemize}
    \item Datasets were loaded using \texttt{pd.read\_csv()}, validated with \texttt{df.info()} and \texttt{df.describe()}, and checked for missing values via \texttt{df.isna().sum()}.
    \item Variables were standardized, datetime fields were created using \texttt{pd.to\_datetime()}, and datasets were merged through index-based joins.
\end{itemize}

\subsubsection*{Feature engineering}
\begin{itemize}
    \item Interest rate and inflation differentials were computed to capture relative monetary conditions and purchasing power, reducing multicollinearity and aligning with interest parity and PPP theories.
    \item The Philippine trade balance was included to represent real-sector foreign exchange flows and macroeconomic shocks.
\end{itemize}

\subsubsection*{Analytical rationale}
\begin{itemize}
    \item These steps produce a concise, economically grounded dataset that captures financial, price-level, and real economic drivers of USD--PHP movements for exploratory data analysis.
\end{itemize}

\subsection{Methods}

\begin{itemize}
    \item The study utilized an \textbf{exploratory data analysis} approach on multi-source time-series datasets, incorporating data cleaning, type conversion, temporal indexing, and alignment through resampling and merging procedures.
    \item \textbf{Descriptive and exploratory techniques} were applied, including missing-data assessment, feature engineering, rolling-window statistics, and decomposition methods to isolate trend, seasonal, and irregular components.
    \item \textbf{Visual time-series analytics} (e.g., line plots and comparative trend charts) were employed to detect temporal patterns, co-movements, and structural variations without enforcing a predefined econometric model.
\end{itemize}

\section{Results}

\subsection{The USD–PHP exchange rate during the pandemic period}

\begin{figure}[htbp]
    \centering
    \includegraphics[width=0.85\textwidth]{images/figure1.png}
    \caption{Philippine Peso (PHP) Exchange Rate Against the US Dollar (USD): Impacts of Global Events (2018--2025)}
    \label{Figure 1}
\end{figure}

\begin{figure}[htbp]
    \centering
    \includegraphics[width=0.85\textwidth]{images/figure2.png}
    \caption{Comparative Inflation of Philippines and USA Response to Global Economic Shocks (2018-2025)}
    \label{Figure 2}
\end{figure}

\section{Validation}

\section{Limitations, Ethics, and Responsible Use}

\section{Conclusion and Actionable Recommendations}

\section*{References}

\section*{Appendix}





\end{document}
