\documentclass[12pt]{article}

\usepackage[utf8]{inputenc}
\usepackage{geometry}
\usepackage{graphicx}
\usepackage{titling}
\usepackage[colorlinks=false, linkcolor=blue, urlcolor=blue, citecolor=blue]{hyperref}
\geometry{margin=1in}
\usepackage{caption}
\captionsetup{justification=centering,font=scriptsize}
\bibliographystyle{unsrt}
\usepackage{float}


% --- Remove default title spacing ---
\setlength{\droptitle}{-2em}  % adjust vertical spacing if needed

% --- Title and author manually ---
\title{\huge\bfseries Temporal Dynamics and Volatility Analysis of the PHP--USD Exchange Rate}
\author{
\footnotesize Baul, K.L., Ca\~{n}ete, M.C., Galario, A., Lapad, G.M., Panilag, A.J., Ramirez, G.M.\\[0.3em]
\footnotesize Department of Data Science\\
\footnotesize University of Science and Technology of Southern Philippines -- CDO
}

\date{}

\begin{document}

% --- Custom title layout ---
\begin{center}
    \includegraphics[width=0.16\textwidth]{logo.png}\par\vspace{1em}
    {\huge\bfseries Temporal Dynamics and Volatility Analysis of the PHP--USD Exchange Rate}\par\vspace{1em}
    {\footnotesize
    Baul, K.L., Ca\~{n}ete, M.C., Galario, A., Lapad, G.M., Panilag, A.J., Ramirez, G.M.\\
    Department of Data Science\\
    University of Science and Technology of Southern Philippines -- CDO\\[0.5em]
    }
\end{center}

\begin{center}
\textbf{Abstract}
\end{center}

\
This study analyzes the behavior, volatility, and structural shifts of the PHP–USD exchange rate from 2018 to 2025. The exchange rate is treated as a dynamic indicator of broader macroeconomic conditions rather than an isolated variable. The analysis integrates inflation and interest rate differentials between the Philippines and the United States, along with trade balance measures, oil prices, and Overseas Filipino Worker (OFW) remittances. Daily USD–PHP closing prices serve as the primary dataset, while monthly macroeconomic indicators are temporally aligned through resampling and feature engineering.

Descriptive statistics, rolling-window measures, and visual time-series techniques are employed to examine trends, volatility patterns, and co-movements without imposing strict econometric assumptions. The results identify distinct phases of exchange rate behavior marked by stability, heightened volatility, and sharp directional changes. Comparative inflation analysis reveals higher variance, faster mean reversion, and weaker persistence in Philippine inflation relative to the smoother and more stable U.S. inflation process. Overall, the findings underscore the sensitivity of the PHP–USD exchange rate to relative macroeconomic pressures and structural dynamics, offering insights relevant to macroeconomic monitoring and risk analysis.

\section{Introduction}

\subsection{Background and Significance}

The PHP--USD exchange rate is a preeminent macroeconomic indicator for the Philippines, a small, open economy heavily reliant on remittances and imports. Fluctuations in the exchange rate directly affect domestic inflation—particularly through the cost of imported commodities such as oil—fiscal stability via US dollar--denominated debt servicing, and the welfare of remittance-receiving households.

Given these critical implications, understanding exchange rate dynamics requires moving beyond a univariate time-series perspective. Classical currency valuation theories, notably Purchasing Power Parity (PPP) and Interest Rate Parity (IRP), posit that exchange rates are fundamentally influenced by cross-country differentials in inflation and interest rates. Consistent with these frameworks, this study incorporates Philippine inflation, US inflation, and the US Federal Funds Rate (DFF) to examine their relationship with the PHP--USD exchange rate.

The 2018–2025 period is particularly salient, as it encompasses extreme macro-financial shocks, including the COVID-19 pandemic (2020) and the aggressive US Federal Reserve interest rate tightening cycle (2022–2023). These events generated pronounced monetary policy divergence, significantly shaping capital flows and exchange rate volatility. Using a multivariate exploratory data analysis (EDA), this study investigates co-movement patterns and volatility dynamics among these variables, providing empirical insights relevant to exchange rate policy and financial risk management.

\subsection{Statement of the Problem}

From 2018 to 2025, the USD–PHP exchange rate experienced significant fluctuations driven by major global and domestic events, including the COVID‑19 pandemic, shifts in interest rates and inflation, oil price instability, and geopolitical tensions such as the Russia–Ukraine war. These events likely affected currency behavior, but the extent, patterns, and timing of their influence remain unclear. This project aims to conduct exploratory data analysis (EDA) to examine how the USD–PHP exchange rate behaved during this period, identify potential seasonal trends, measure changes in volatility before and after the pandemic, and assess whether major events or macroeconomic factors contributed to notable movements in the exchange rate. Understanding these patterns can provide insights into currency sensitivity to shocks and support more informed economic analysis.

\section{Methodology}

\subsection{Primary Datasets}

This research employs the USD--PHP exchange rate as its primary time-series dataset. We sourced the USD--PHP exchange rate dataset from \href{https://www.investing.com/currencies/usd-php}{Investing.com (2025)}. To maintain a parsimonious model and ensure data stationarity, we extracted only the Daily Closing Price, omitting the auxiliary Open, High, and Low (OHL) metrics. The rationale for this selection is twofold: first, the closing price represents the aggregate daily market consensus; and second, it minimizes volatility bias inherent in intraday trading. This allows for a more robust correlation analysis with secondary datasets that lack high-frequency, intraday granularity.

\subsection{Secondary Datasets}

To provide a comprehensive analysis of the USD-PHP exchange rate, this study integrates a primary exchange rate dataset with four distinct categories of secondary macroeconomic variables. These variables represent the fundamental drivers of currency valuation, encompassing both financial flows and real-economy demands.

First, the study incorporates \textbf{Interest Rates and Inflation Rates} from both the United States and the Philippines. These are utilized to capture the \textbf{'interest rate differential'} and \textbf{'purchasing power parity'} effects, which dictate capital allocation and investment flows between the two nations.

Second, to account for the structural supply-demand dynamics of the Philippine economy, the study integrates \textbf{Merchandise Trade Statistics}—specifically the total value of \textbf{Merchandise Imports and Exports} for the Philippines. In the Philippine context, the \textbf{Merchandise Trade Balance} is a critical determinant of currency value; a widening trade deficit in physical goods necessitates a higher demand for US Dollars to settle international payments, typically leading to Peso depreciation. By focusing on merchandise trade, the analysis captures the direct impact of the physical economy on foreign exchange requirements.

Furthermore, as the Philippines is a net energy importer, international \textbf{Oil Prices (Crude and Brent)} are included to reflect the cost-push inflationary pressures and the subsequent drain on foreign exchange reserves required for fuel procurement. Finally, \textbf{OFW Remittances} are integrated as a primary supply-side factor. As a significant contributor to the Philippine GDP, these remittances represent a consistent inflow of US Dollars, acting as a key stabilizer for the local currency against external shocks. By combining these datasets, the model can effectively distinguish between short-term financial speculation and the fundamental commercial requirements of the Philippine economy.

\subsection{Data Cleaning and Preprocessing}

\subsubsection*{Data preparation}
\begin{itemize}
    \item Datasets were loaded using \texttt{pd.read\_csv()}, validated with \texttt{df.info()} and \texttt{df.describe()}, and checked for missing values via \texttt{df.isna().sum()}.
    \item Variables were standardized, datetime fields were created using \texttt{pd.to\_datetime()}, and datasets were merged through index-based joins.
\end{itemize}

\subsubsection*{Feature engineering}
\begin{itemize}
    \item Interest rate and inflation differentials were computed to capture relative monetary conditions and purchasing power, reducing multicollinearity and aligning with interest parity and PPP theories.
    \item The Philippine trade balance was included to represent real-sector foreign exchange flows and macroeconomic shocks.
\end{itemize}

\subsubsection*{Analytical rationale}
\begin{itemize}
    \item These steps produce a concise, economically grounded dataset that captures financial, price-level, and real economic drivers of USD--PHP movements for exploratory data analysis.
\end{itemize}

\subsection{Methods}

\begin{itemize}
    \item The study utilized an \textbf{exploratory data analysis} approach on multi-source time-series datasets, incorporating data cleaning, type conversion, temporal indexing, and alignment through resampling and merging procedures.
    \item \textbf{Descriptive and exploratory techniques} were applied, including missing-data assessment, feature engineering, rolling-window statistics, and decomposition methods to isolate trend, seasonal, and irregular components.
    \item \textbf{Visual time-series analytics} (e.g., line plots and comparative trend charts) were employed to detect temporal patterns, co-movements, and structural variations without enforcing a predefined econometric model.
\end{itemize}

\section{Results}

\subsection{The USD–PHP exchange rate during the pandemic period}

\begin{figure}[H]
    \centering
    \includegraphics[width=0.85\textwidth]{images/figure1.png}
    \caption{Philippine Peso (PHP) Exchange Rate Against the US Dollar (USD): Impacts of Global Events (2018--2025)}
    \label{Figure 1}
\end{figure}

Between Q1 2020 and Q4 2021, the PHP exhibited notable strength, bottoming out near 48.00 PHP/USD. However, the onset of aggressive US Federal Reserve tightening in January 2022, compounded by heightened geopolitical risk from the Russia-Ukraine conflict, triggered a sharp reversal. By late 2022, the Peso reached historic lows near 59.00 PHP/USD. While a brief recovery followed in early 2023, the exchange rate has since stabilized within a volatile range of 54.00 to 58.00 PHP/USD, reflecting persistent inflationary pressures and a widened trade deficit.

\begin{figure}[H]
    \centering
    \includegraphics[width=0.80\textwidth]{images/figure1.2.png}
    \caption{USD-PHP FX Volatility: Heat Timeline Highlighting Start to End of Pandemic}
    \label{Figure 2}
\end{figure}

The Philippine Peso (PHP) exhibited pronounced volatility from 2018 to 2025, as reflected in the green-shaded FX timeline. Darker green shades correspond to periods of higher month-to-month exchange rate swings, while lighter green indicates calmer, more stable periods. From 2018 to 2019, the Peso remained relatively stable, with light to medium shades signaling a low-volatility environment. In mid-2021, volatility surged sharply, marking the first major peak of the period, likely driven by domestic inflationary pressures and heightened global uncertainty during the pandemic recovery phase. After this initial peak, volatility eased somewhat but remained above pre-2020 levels, reflecting lingering economic uncertainty. Toward the end of 2022, the Peso experienced a second, more intense spike in volatility, coinciding with aggressive U.S. Federal Reserve interest rate hikes and surges in global inflation. Just before the end of the pandemic period in May 2023, volatility reached an extreme rate of 0.1, reflecting a combination of residual pandemic effects, trade imbalances, and speculative FX activity. In 2024, volatility moderated but began to rise slightly toward year-end, indicating renewed external and domestic FX pressures, highlighting the Peso’s continued sensitivity to both global shocks and structural economic factors.

\begin{figure}[H]
    \centering
    \includegraphics[width=0.85\textwidth]{images/figure2.png}
    \caption{Comparative Inflation of Philippines and USA Response to Global Economic Shocks (2018-2025)}
    \label{Figure 3}
\end{figure}

The graph shows that the Philippines (PH) inflation series has a much wider spread of values than the United States (US) series. This means PH inflation fluctuates more around its average, indicating higher variance and standard deviation, while US inflation remains more tightly clustered. PH inflation is also more volatile, as shown by its jagged line and frequent sharp month-to-month changes. In contrast, the US line is smoother and more gradual, reflecting lower short-term volatility. Additionally, the US series shows stronger trend persistence, where increases or decreases last longer, implying higher autocorrelation, while the PH series changes direction more frequently.

In terms of peaks and troughs, PH inflation has sharp, short-lived highs and lows followed by quick reversals, which indicates fast mean reversion. The US series displays broader peaks and slower declines, meaning deviations from the average persist longer. The PH series also tends to reach turning points earlier than the US series, suggesting a lead–lag pattern where PH adjusts faster. Moreover, the PH line has steeper slopes, showing that changes occur more rapidly, while the US series changes more gradually. Overall, the graph shows that PH behaves as a high-volatility, fast-moving process, whereas US behaves as a more stable, low-volatility, and persistent process over time.

\begin{figure}[H]
    \centering
    \includegraphics[width=0.85\textwidth]{images/figure3.png}
    \caption{ Trade Deficit and Pandemic Impact on Philippine Goods Flow (2018–2025)}
    \label{Figure 4}
\end{figure}

The sustained \textbf{trade deficit} shown in Figure 4, characterized by a worsening trend from 2021 to 2023, is fundamentally explained by the divergent patterns in exports and imports visible in Graph~\ref{app:graph:6}. The latter shows that, while \textbf{Exports} (blue line) displayed only moderate growth after the initial pandemic dip in early 2020, \textbf{Imports} (green line) experienced a robust, significant, and persistent recovery, consistently maintaining a value substantially higher than exports throughout the period. This dominant import growth necessitates a constant outflow of US Dollars, which is structurally reflected in the persistent and volatile trade deficits of Figure 3. Global commodity costs compound this pressure: Graph~\ref{app:graph:10} reveals a \textbf{high negative correlation} between the \textbf{Trade Balance} and \textbf{Price Crude} (Crude Oil) and \textbf{Price Brent} (Brent Oil) (values around $-0.83$), indicating that when global oil prices rise (as seen in the peak in Graph~\ref{app:graph:2}), the trade balance deteriorates sharply. Since the Philippines is a net energy importer, higher oil prices directly increase the cost of its already high import bill, requiring more foreign currency to settle payments and worsening the trade deficit, a key factor that puts downward pressure on the Peso. Ultimately, the trade balance's \textbf{high negative correlation} with the \textbf{Price USD-PHP} (around $-0.61$ in Graph~\ref{app:graph:10}) confirms that the worsening deficit strongly drives the depreciation (increase in the value) of the Peso.

\begin{figure}[H]
    \centering
    \includegraphics[width=0.85\textwidth]{images/figure4.png}
    \caption{Correlation between US-PH Interest Rate Differences and the Peso's Value}
    \label{Figure 5}
\end{figure}

The scatter plot in Figure 5 visually confirms a \textbf{strong, positive linear correlation} between the Interest Rate Differential (US Rate minus PH Rate) on the horizontal axis and the USD-PHP Exchange Rate on the vertical axis. This key feature, represented by the upward-sloping regression line and tightly clustered data points, is statistically supported by Graph~\ref{app:graph:10} (Heatmap), which shows a \textbf{strong positive correlation} between the \textbf{Interest Differential} and the \textbf{Price USD-PHP} (a value around $+0.65$). This relationship implies that as the Interest Rate Differential moves to the right (becoming less negative, meaning the US interest rate is declining relative to the Philippine rate), the USD-PHP Exchange Rate tends to increase, causing the Philippine Peso to depreciate. Conversely, when the Interest Differential becomes more negative (as the US rate rises relative to the PH rate), the Peso tends to appreciate. The overall tightness of the data in Figure 4, reinforced by the high correlation coefficient in Graph~\ref{app:graph:10}, suggests that the divergence in interest rates between the two countries is a \textbf{significant and reliable predictor} of the movements in the USD-PHP exchange rate, directly supporting the principles of Interest Rate Parity theory.

\begin{figure}[H]
    \centering
    \includegraphics[width=0.85\textwidth]{images/figure5.png}
    \caption{USD-PHP Exchange Rate Volatility: Pinpointing Dates of Major Trend Changes}
    \label{Figure 6}
\end{figure}

The structural \textbf{Change Point Break Dates} identified in Figure 6 serve as markers for fundamental shifts in the USD-PHP exchange rate's behavior, which are comprehensively supported by corresponding movements in underlying macroeconomic drivers (Graphs~\ref{app:graph:3}, \ref{app:graph:4}, \ref{app:graph:5}, \ref{app:graph:7}, \ref{app:graph:10}, \ref{app:graph:11}, and \ref{app:graph:13}). The overall narrative of the exchange rate shifting from initial stability to \textbf{significant depreciation} is directly tied to changes in cross-country differentials (Graph~\ref{app:graph:13}). The \textbf{strong positive correlation} between the \textbf{Interest Differential} and the \textbf{Price USD-PHP} (around $+0.65$ in Graph~\ref{app:graph:10}, and visually confirmed in Figure 5 and Graph~\ref{app:graph:11}) confirms that shifts in interest rate policy are the primary structural mechanism driving the Peso's value.

This mechanism is evident in two major phases:

\begin{enumerate}
    \item \textbf{Break 2 (Nov 2021) and the Depreciation Phase:} This break precedes the \textbf{Start of Fed Hikes (Jan 2022)} and marks the end of the Peso's post-pandemic strength. It coincides with the \textbf{Interest Differential (Graph~\ref{app:graph:3})} nearing its least negative point and the \textbf{Inflation Differential (Graph~\ref{app:graph:4})} beginning its sharp spike. Crucially, the \textbf{Multivariate Change Point Detection (Graph~\ref{app:graph:13})} highlights this as the beginning of a sustained macro regime change (blue-to-pink transition in late 2021/early 2022) across the Exchange Rate, Interest Differential, and Inflation Differential. This shift was fueled by the strong negative correlation between the \textbf{Trade Balance} and the \textbf{Price USD-PHP} ($-0.61$ in Graph~\ref{app:graph:10}), as the \textbf{Trade Balance (Graph~\ref{app:graph:7})} was already rapidly deteriorating.
    
    \item \textbf{Break 3 (Sep 2022) and Maximum Pressure:} This break coincides with the height of the \textbf{Russia-Ukraine invasion (Feb 2022)}, which propelled oil prices to their peak (Graph~\ref{app:graph:2}). The \textbf{Trade Balance (Graph~\ref{app:graph:7})} reached its lowest point shortly after, driven by the strong negative correlation with rising oil costs (Graph~\ref{app:graph:10}). The resulting deficit created maximum demand for US Dollars, coinciding with the Peso's lowest value. This peak pressure occurred despite \textbf{OFW Remittances (Graph~\ref{app:graph:5})} maintaining a stable, cyclical inflow, underscoring that the external pressures from trade and monetary policy vastly outweighed the stabilizing effect of remittances on the foreign currency supply.
\end{enumerate}

In summary, the statistically identified breaks in Figure 5 are not isolated events but rather the result of a \textbf{synchronized regime change} across the Interest Rate Differential, the Trade Balance, and the Inflation Differential, all acting consistently with their high correlation coefficients in Graph~\ref{app:graph:10} to push the Peso into its deepest depreciation phase.

\section{Validation}

The validation of findings was conducted using a rigorous exploratory data analysis framework to ensure data quality, analytical accuracy, and methodological soundness. Data preprocessing procedures were validated through systematic completeness checks, descriptive statistical profiling, and consistency assessments to confirm numerical integrity and temporal alignment. Distributional characteristics were examined using both graphical and statistical techniques to identify outliers, assess dispersion, and confirm economic plausibility.

Relationships among variables were explored using correlation analysis, and the observed associations were directionally and theoretically consistent with established exchange rate and macroeconomic theory. Time-series visualizations further supported the validity of observed trends by confirming temporal continuity, structural shifts, and periods of heightened volatility. Additional inspection of variability and clustering patterns provided preliminary evidence relevant to exchange rate volatility behavior. Extreme observations were evaluated within their historical context and retained when indicative of genuine market shocks rather than data anomalies. The stability of relationships across time intervals further reinforced the reliability of the analytical results. Overall, the consistency of findings across multiple exploratory techniques confirms the robustness of the dataset and its suitability for advanced volatility analysis.

\section{Limitations, Ethics, and Responsible Use}

\subsection{Limitations}

\begin{itemize}
    \item \textbf{Frequency:} Aggregation to monthly data loses high-frequency volatility (daily clustering) that is critical for short-term risk management.
    \item \textbf{Model Scope:} The OLS model in the analysis is simplified (excluding the DFF and other critical variables like trade balance/remittances) and assumes linear relationships.
    \item \textbf{Causality vs. Correlation:} The OLS provides correlation and association, but does not definitively prove causation, especially without formal co-integration testing.
\end{itemize}

\subsection{Ethics and Responsible Use}

Maintain the same ethical standards regarding transparency and the explicit disclaimer that the results are purely \textbf{exploratory and academic}, and not a basis for financial investment decisions

\section{Conclusion and Actionable Recommendations}

\subsection{Conclusion}

This exploratory analysis confirms that the USD–PHP exchange rate is a dynamic indicator highly sensitive to structural and global macroeconomic shifts. The findings establish three key relationships driving the currency's behavior:

\begin{enumerate} 

    \item \textbf{High External Sensitivity:} The Peso's trajectory shifted sharply in response to major external events, notably the Start of Fed Hikes (Jan 2022), when the U.S. Federal Reserve 
    signaled a major policy tightening \cite{TheStreetFed2025}, which triggered a sustained depreciation 
    to historic highs.
    
    \item \textbf{Structural Predictors:} The analysis validates key theoretical drivers:
    \begin{itemize}
        \item A strong, positive linear correlation exists between the Interest Rate Differential (US minus PH) 
        and the USD–PHP exchange rate, confirming its predictive reliability consistent with Interest Rate Parity.
        \item The persistent and structurally large trade deficit acts as an ongoing depreciation pressure, 
        requiring high US Dollar demand for imports.
    \end{itemize}
    
    \item \textbf{Domestic Volatility: Philippine inflation} is fundamentally a \textbf{high-volatility, fast-moving process} 
    with high variance and fast mean reversion, contrasting with the more stable U.S. inflation. 
    This intrinsic instability contributes to the high volatility observed in the Peso.
\end{enumerate}

In conclusion, the results underscore the need for policymakers to manage the structural pressures 
of the trade deficit and anticipate the swift, powerful impact of U.S. monetary policy divergence 
for effective macroeconomic monitoring and currency risk management.

\subsection{Actionable Recommendations}

\begin{itemize}
    \item \textbf{Adopt a multifaceted policy response} - Address both short-term financial stability and long-term structural resilience to reduce external vulnerability.
    \item \textbf{Implement Proactive Differential Management (BSP)} - Actively use monetary policy tools to maintain favorable interest rate differentials relative to U.S. rate hikes, mitigating interest rate parity (IRP)–driven peso depreciation.
    \item \textbf{Apply Strategic Commodity Hedging} - Introduce targeted hedging mechanisms for oil imports to protect the economy and foreign exchange reserves from cost-push inflation and external price volatility.
    \item \textbf{Introduce targeted hedging mechanisms for oil imports to protect the economy and foreign exchange reserves from cost-push inflation and external price volatility.} - Promote long-term, high-quality Foreign Direct Investment to generate sustainable USD inflows and counter persistent trade balance pressures.
\end{itemize}

\nocite{*}
\bibliography{references}

\section*{Appendix}
\appendix

\subsection*{Graphs}

\begin{figure}[H]
\centering
\includegraphics[width=0.85\textwidth]{images/graph1.png}
\caption{USD-PHP Monthly Average Exchange Rate (Jan 2018 - Aug 2025)}
\label{Graph 1}
\end{figure}

\begin{figure}[H]
\centering
\includegraphics[width=0.85\textwidth]{images/graph2.png}
\caption{RMonthly Average Oil Price (Jan 2018 - Aug 2025)}
\label{Graph 2}
\end{figure}

\subsection*{Tables}

\begin{figure}[H]
\centering
\includegraphics[width=0.80\textwidth]{images/table1.png}
\caption{Statistical Summary}
\label{Table 1}
\end{figure}

\begin{figure}[H]
\centering
\includegraphics[width=0.50\textwidth]{images/table2.png}
\caption{ VIF Results}
\label{Table 2}
\end{figure}


\newpage
\section*{Artifact links}

\subsection*{Data Sources}

\begin{itemize}
    \item \textbf{GitHub Repository:} 
    \url{https://github.com/kbbn-debugger/Exploratory-Data-Analysis-USD-PHP}

    \item \textbf{USD-PHP Exchange Rate (Jan 2018 -- Dec 2025):} 
    \url{https://ph.investing.com/currencies/usd-php-historical-data}

    \item \textbf{Oil Prices (Jan 2018 -- Dec 2025):} 
    \begin{itemize}
        \item Crude Oil Historical Data: \url{https://ph.investing.com/commodities/crude-oil-historical-data}
        \item Brent Oil Historical Data: \url{https://ph.investing.com/commodities/brent-oil}
    \end{itemize}

    \item \textbf{Total Merchandise of the Philippines – Imports and Exports (Monthly, 2018 -- 2025):}
    \begin{itemize}
        \item Steps to extract:
        \begin{enumerate}
            \item Select ``Total Merchandise Monthly''
            \item Choose ``Philippines'' as Reporting Economy
            \item Select ``Total Merchandise'' in products/sectors
            \item Choose ``World'' as Partner Economies
            \item Select years 2018 to 2025
        \end{enumerate}
        \item Source: \url{https://stats.wto.org/}
    \end{itemize}

    \item \textbf{Inflation Prices:}
    \begin{itemize}
        \item United States (Core Inflation): \url{https://fred.stlouisfed.org/series/FPCPITOTLZGUSA}
        \item Philippines (Core Inflation): \url{https://www.bsp.gov.ph/SitePages/Statistics/Prices.aspx?TabId=1}
    \end{itemize}

    \item \textbf{Interest Rates:}
    \begin{itemize}
        \item Federal Funds Effective Rate (US): \url{https://fred.stlouisfed.org/series/DFF}
        \item Daily Overnight Reverse Repurchase (RRP, PH): \url{https://www.bsp.gov.ph/SitePages/Statistics/Financial%20System%20Accounts.aspx?TabId=14}
    \end{itemize}
\end{itemize}




\end{document}
