\documentclass[12pt]{article}

\usepackage[utf8]{inputenc}
\usepackage{geometry}
\usepackage{graphicx}
\usepackage{titling}
\usepackage[colorlinks=false, linkcolor=blue, urlcolor=blue, citecolor=blue]{hyperref}
\geometry{margin=1in}
\usepackage{caption}
\captionsetup{justification=centering,font=scriptsize}
\bibliographystyle{unsrt}


% --- Remove default title spacing ---
\setlength{\droptitle}{-2em}  % adjust vertical spacing if needed

% --- Title and author manually ---
\title{\huge\bfseries Temporal Dynamics and Volatility Analysis of the PHP--USD Exchange Rate}
\author{
\footnotesize Baul, K.L., Ca\~{n}ete, M.C., Galario, A., Lapad, G.M., Panilag, A.J., Ramirez, G.M.\\[0.3em]
\footnotesize Department of Data Science\\
\footnotesize University of Science and Technology of Southern Philippines -- CDO
}

\date{}

\begin{document}

% --- Custom title layout ---
\begin{center}
    \includegraphics[width=0.16\textwidth]{logo.png}\par\vspace{1em}
    {\huge\bfseries Temporal Dynamics and Volatility Analysis of the PHP--USD Exchange Rate}\par\vspace{1em}
    {\footnotesize
    Baul, K.L., Ca\~{n}ete, M.C., Galario, A., Lapad, G.M., Panilag, A.J., Ramirez, G.M.\\
    Department of Data Science\\
    University of Science and Technology of Southern Philippines -- CDO\\[0.5em]
    }
\end{center}

\begin{center}
\textbf{Abstract}
\end{center}

\
This study analyzes the behavior, volatility, and structural shifts of the PHP–USD exchange rate from 2018 to 2025. The exchange rate is treated as a dynamic indicator of broader macroeconomic conditions rather than an isolated variable. The analysis integrates inflation and interest rate differentials between the Philippines and the United States, along with trade balance measures, oil prices, and Overseas Filipino Worker (OFW) remittances. Daily USD–PHP closing prices serve as the primary dataset, while monthly macroeconomic indicators are temporally aligned through resampling and feature engineering.

Descriptive statistics, rolling-window measures, and visual time-series techniques are employed to examine trends, volatility patterns, and co-movements without imposing strict econometric assumptions. The results identify distinct phases of exchange rate behavior marked by stability, heightened volatility, and sharp directional changes. Comparative inflation analysis reveals higher variance, faster mean reversion, and weaker persistence in Philippine inflation relative to the smoother and more stable U.S. inflation process. Overall, the findings underscore the sensitivity of the PHP–USD exchange rate to relative macroeconomic pressures and structural dynamics, offering insights relevant to macroeconomic monitoring and risk analysis.

\section{Introduction}

\subsection{Background and Significance}

The PHP--USD exchange rate is a preeminent macroeconomic indicator for the Philippines, a small, open economy heavily reliant on remittances and imports. Fluctuations in the exchange rate directly affect domestic inflation—particularly through the cost of imported commodities such as oil—fiscal stability via US dollar--denominated debt servicing, and the welfare of remittance-receiving households.

Given these critical implications, understanding exchange rate dynamics requires moving beyond a univariate time-series perspective. Classical currency valuation theories, notably Purchasing Power Parity (PPP) and Interest Rate Parity (IRP), posit that exchange rates are fundamentally influenced by cross-country differentials in inflation and interest rates. Consistent with these frameworks, this study incorporates Philippine inflation, US inflation, and the US Federal Funds Rate (DFF) to examine their relationship with the PHP--USD exchange rate.

The 2018–2025 period is particularly salient, as it encompasses extreme macro-financial shocks, including the COVID-19 pandemic (2020) and the aggressive US Federal Reserve interest rate tightening cycle (2022–2023). These events generated pronounced monetary policy divergence, significantly shaping capital flows and exchange rate volatility. Using a multivariate exploratory data analysis (EDA), this study investigates co-movement patterns and volatility dynamics among these variables, providing empirical insights relevant to exchange rate policy and financial risk management.

\subsection{Statement of the Problem}

From 2018 to 2025, the USD–PHP exchange rate experienced significant fluctuations driven by major global and domestic events, including the COVID‑19 pandemic, shifts in interest rates and inflation, oil price instability, and geopolitical tensions such as the Russia–Ukraine war. These events likely affected currency behavior, but the extent, patterns, and timing of their influence remain unclear. This project aims to conduct exploratory data analysis (EDA) to examine how the USD–PHP exchange rate behaved during this period, identify potential seasonal trends, measure changes in volatility before and after the pandemic, and assess whether major events or macroeconomic factors contributed to notable movements in the exchange rate. Understanding these patterns can provide insights into currency sensitivity to shocks and support more informed economic analysis.

\section{Methodology}

\subsection{Primary Datasets}

This research employs the USD--PHP exchange rate as its primary time-series dataset. We sourced the USD--PHP exchange rate dataset from \href{https://www.investing.com/currencies/usd-php}{Investing.com (2025)}. To maintain a parsimonious model and ensure data stationarity, we extracted only the Daily Closing Price, omitting the auxiliary Open, High, and Low (OHL) metrics. The rationale for this selection is twofold: first, the closing price represents the aggregate daily market consensus; and second, it minimizes volatility bias inherent in intraday trading. This allows for a more robust correlation analysis with secondary datasets that lack high-frequency, intraday granularity.

\subsection{Secondary Datasets}

To provide a comprehensive analysis of the USD-PHP exchange rate, this study integrates a primary exchange rate dataset with four distinct categories of secondary macroeconomic variables. These variables represent the fundamental drivers of currency valuation, encompassing both financial flows and real-economy demands.

First, the study incorporates Interest Rates and Inflation Rates from both the United States and the Philippines. These are utilized to capture the 'interest rate differential' and 'purchasing power parity' effects, which dictate capital allocation and investment flows between the two nations.

Second, to account for the structural supply-demand dynamics of the Philippine economy, the study integrates Merchandise Trade Statistics—specifically the total value of Merchandise Imports and Exports for the Philippines. In the Philippine context, the Merchandise Trade Balance is a critical determinant of currency value; a widening trade deficit in physical goods necessitates a higher demand for US Dollars to settle international payments, typically leading to Peso depreciation. By focusing on merchandise trade, the analysis captures the direct impact of the physical economy on foreign exchange requirements.

Furthermore, as the Philippines is a net energy importer, international Oil Prices (Crude and Brent) are included to reflect the cost-push inflationary pressures and the subsequent drain on foreign exchange reserves required for fuel procurement. Finally, OFW Remittances are integrated as a primary supply-side factor. As a significant contributor to the Philippine GDP, these remittances represent a consistent inflow of US Dollars, acting as a key stabilizer for the local currency against external shocks. By combining these datasets, the model can effectively distinguish between short-term financial speculation and the fundamental commercial requirements of the Philippine economy.

\subsection{Data Cleaning and Preprocessing}

\subsubsection*{Data preparation}
\begin{itemize}
    \item Datasets were loaded using \texttt{pd.read\_csv()}, validated with \texttt{df.info()} and \texttt{df.describe()}, and checked for missing values via \texttt{df.isna().sum()}.
    \item Variables were standardized, datetime fields were created using \texttt{pd.to\_datetime()}, and datasets were merged through index-based joins.
\end{itemize}

\subsubsection*{Feature engineering}
\begin{itemize}
    \item Interest rate and inflation differentials were computed to capture relative monetary conditions and purchasing power, reducing multicollinearity and aligning with interest parity and PPP theories.
    \item The Philippine trade balance was included to represent real-sector foreign exchange flows and macroeconomic shocks.
\end{itemize}

\subsubsection*{Analytical rationale}
\begin{itemize}
    \item These steps produce a concise, economically grounded dataset that captures financial, price-level, and real economic drivers of USD--PHP movements for exploratory data analysis.
\end{itemize}

\subsection{Methods}

\begin{itemize}
    \item The study utilized an \textbf{exploratory data analysis} approach on multi-source time-series datasets, incorporating data cleaning, type conversion, temporal indexing, and alignment through resampling and merging procedures.
    \item \textbf{Descriptive and exploratory techniques} were applied, including missing-data assessment, feature engineering, rolling-window statistics, and decomposition methods to isolate trend, seasonal, and irregular components.
    \item \textbf{Visual time-series analytics} (e.g., line plots and comparative trend charts) were employed to detect temporal patterns, co-movements, and structural variations without enforcing a predefined econometric model.
\end{itemize}

\section{Results}

\subsection{The USD–PHP exchange rate during the pandemic period}

\begin{figure}[htbp]
    \centering
    \includegraphics[width=0.85\textwidth]{images/figure1.png}
    \caption{Philippine Peso (PHP) Exchange Rate Against the US Dollar (USD): Impacts of Global Events (2018--2025)}
    \label{Figure 1}
\end{figure}

Between Q1 2020 and Q4 2021, the PHP exhibited notable strength, bottoming out near 48.00 PHP/USD. However, the onset of aggressive US Federal Reserve tightening in January 2022, compounded by heightened geopolitical risk from the Russia-Ukraine conflict, triggered a sharp reversal. By late 2022, the Peso reached historic lows near 59.00 PHP/USD. While a brief recovery followed in early 2023, the exchange rate has since stabilized within a volatile range of 54.00 to 58.00 PHP/USD, reflecting persistent inflationary pressures and a widened trade deficit.

\begin{figure}[htbp]
    \centering
    \includegraphics[width=0.85\textwidth]{images/figure2.png}
    \caption{Comparative Inflation of Philippines and USA Response to Global Economic Shocks (2018-2025)}
    \label{Figure 2}
\end{figure}



\section{Validation}

The validation of findings was conducted using a rigorous exploratory data analysis framework to ensure data quality, analytical accuracy, and methodological soundness. Data preprocessing procedures were validated through systematic completeness checks, descriptive statistical profiling, and consistency assessments to confirm numerical integrity and temporal alignment. Distributional characteristics were examined using both graphical and statistical techniques to identify outliers, assess dispersion, and confirm economic plausibility.

Relationships among variables were explored using correlation analysis, and the observed associations were directionally and theoretically consistent with established exchange rate and macroeconomic theory. Time-series visualizations further supported the validity of observed trends by confirming temporal continuity, structural shifts, and periods of heightened volatility. Additional inspection of variability and clustering patterns provided preliminary evidence relevant to exchange rate volatility behavior. Extreme observations were evaluated within their historical context and retained when indicative of genuine market shocks rather than data anomalies. The stability of relationships across time intervals further reinforced the reliability of the analytical results. Overall, the consistency of findings across multiple exploratory techniques confirms the robustness of the dataset and its suitability for advanced volatility analysis.

\section{Limitations, Ethics, and Responsible Use}

\subsection{Limitations}

\begin{itemize}
    \item \textbf{Frequency:} Aggregation to monthly data loses high-frequency volatility (daily clustering) that is critical for short-term risk management.
    \item \textbf{Model Scope:} The OLS model in the analysis is simplified (excluding the DFF and other critical variables like trade balance/remittances) and assumes linear relationships.
    \item \textbf{Causality vs. Correlation:} The OLS provides correlation and association, but does not definitively prove causation, especially without formal co-integration testing.
\end{itemize}

\subsection{Ethics and Responsible Use}

Maintain the same ethical standards regarding transparency and the explicit disclaimer that the results are purely \textbf{exploratory and academic}, and not a basis for financial investment decisions

\section{Conclusion and Actionable Recommendations}

\subsection{Conclusion}

This exploratory analysis confirms that the USD–PHP exchange rate is a dynamic indicator highly sensitive to structural and global macroeconomic shifts. The findings establish three key relationships driving the currency's behavior:

\begin{enumerate} 

    \item \textbf{High External Sensitivity:} The Peso's trajectory shifted sharply in response to major external events, notably the Start of Fed Hikes (Jan 2022), when the U.S. Federal Reserve 
    signaled a major policy tightening \cite{TheStreetFed2025}, which triggered a sustained depreciation 
    to historic highs.
    
    \item \textbf{Structural Predictors:} The analysis validates key theoretical drivers:
    \begin{itemize}
        \item A strong, positive linear correlation exists between the Interest Rate Differential (US minus PH) 
        and the USD–PHP exchange rate, confirming its predictive reliability consistent with Interest Rate Parity.
        \item The persistent and structurally large trade deficit acts as an ongoing depreciation pressure, 
        requiring high US Dollar demand for imports.
    \end{itemize}
    
    \item \textbf{Domestic Volatility: Philippine inflation} is fundamentally a \textbf{high-volatility, fast-moving process} 
    with high variance and fast mean reversion, contrasting with the more stable U.S. inflation. 
    This intrinsic instability contributes to the high volatility observed in the Peso.
\end{enumerate}

In conclusion, the results underscore the need for policymakers to manage the structural pressures 
of the trade deficit and anticipate the swift, powerful impact of U.S. monetary policy divergence 
for effective macroeconomic monitoring and currency risk management.

\subsection{Actionable Recommendations}

\begin{itemize}
    \item \textbf{Adopt a multifaceted policy response} - Address both short-term financial stability and long-term structural resilience to reduce external vulnerability.
    \item \textbf{Implement Proactive Differential Management (BSP)} - Actively use monetary policy tools to maintain favorable interest rate differentials relative to U.S. rate hikes, mitigating interest rate parity (IRP)–driven peso depreciation.
    \item \textbf{Apply Strategic Commodity Hedging} - Introduce targeted hedging mechanisms for oil imports to protect the economy and foreign exchange reserves from cost-push inflation and external price volatility.
    \item \textbf{Introduce targeted hedging mechanisms for oil imports to protect the economy and foreign exchange reserves from cost-push inflation and external price volatility.} - Promote long-term, high-quality Foreign Direct Investment to generate sustainable USD inflows and counter persistent trade balance pressures.
\end{itemize}

\nocite{*}
\bibliography{references}

\section*{Appendix}





\end{document}
